% Options for packages loaded elsewhere
\PassOptionsToPackage{unicode}{hyperref}
\PassOptionsToPackage{hyphens}{url}
%
\documentclass[
]{article}
\usepackage{lmodern}
\usepackage{amssymb,amsmath}
\usepackage{ifxetex,ifluatex}
\ifnum 0\ifxetex 1\fi\ifluatex 1\fi=0 % if pdftex
  \usepackage[T1]{fontenc}
  \usepackage[utf8]{inputenc}
  \usepackage{textcomp} % provide euro and other symbols
\else % if luatex or xetex
  \usepackage{unicode-math}
  \defaultfontfeatures{Scale=MatchLowercase}
  \defaultfontfeatures[\rmfamily]{Ligatures=TeX,Scale=1}
\fi
% Use upquote if available, for straight quotes in verbatim environments
\IfFileExists{upquote.sty}{\usepackage{upquote}}{}
\IfFileExists{microtype.sty}{% use microtype if available
  \usepackage[]{microtype}
  \UseMicrotypeSet[protrusion]{basicmath} % disable protrusion for tt fonts
}{}
\makeatletter
\@ifundefined{KOMAClassName}{% if non-KOMA class
  \IfFileExists{parskip.sty}{%
    \usepackage{parskip}
  }{% else
    \setlength{\parindent}{0pt}
    \setlength{\parskip}{6pt plus 2pt minus 1pt}}
}{% if KOMA class
  \KOMAoptions{parskip=half}}
\makeatother
\usepackage{xcolor}
\IfFileExists{xurl.sty}{\usepackage{xurl}}{} % add URL line breaks if available
\IfFileExists{bookmark.sty}{\usepackage{bookmark}}{\usepackage{hyperref}}
\hypersetup{
  pdftitle={Assignment 1 ETC1010 - 5510},
  pdfauthor={Jason Ching Yuen Siu},
  hidelinks,
  pdfcreator={LaTeX via pandoc}}
\urlstyle{same} % disable monospaced font for URLs
\usepackage[margin=1in]{geometry}
\usepackage{color}
\usepackage{fancyvrb}
\newcommand{\VerbBar}{|}
\newcommand{\VERB}{\Verb[commandchars=\\\{\}]}
\DefineVerbatimEnvironment{Highlighting}{Verbatim}{commandchars=\\\{\}}
% Add ',fontsize=\small' for more characters per line
\usepackage{framed}
\definecolor{shadecolor}{RGB}{248,248,248}
\newenvironment{Shaded}{\begin{snugshade}}{\end{snugshade}}
\newcommand{\AlertTok}[1]{\textcolor[rgb]{0.94,0.16,0.16}{#1}}
\newcommand{\AnnotationTok}[1]{\textcolor[rgb]{0.56,0.35,0.01}{\textbf{\textit{#1}}}}
\newcommand{\AttributeTok}[1]{\textcolor[rgb]{0.77,0.63,0.00}{#1}}
\newcommand{\BaseNTok}[1]{\textcolor[rgb]{0.00,0.00,0.81}{#1}}
\newcommand{\BuiltInTok}[1]{#1}
\newcommand{\CharTok}[1]{\textcolor[rgb]{0.31,0.60,0.02}{#1}}
\newcommand{\CommentTok}[1]{\textcolor[rgb]{0.56,0.35,0.01}{\textit{#1}}}
\newcommand{\CommentVarTok}[1]{\textcolor[rgb]{0.56,0.35,0.01}{\textbf{\textit{#1}}}}
\newcommand{\ConstantTok}[1]{\textcolor[rgb]{0.00,0.00,0.00}{#1}}
\newcommand{\ControlFlowTok}[1]{\textcolor[rgb]{0.13,0.29,0.53}{\textbf{#1}}}
\newcommand{\DataTypeTok}[1]{\textcolor[rgb]{0.13,0.29,0.53}{#1}}
\newcommand{\DecValTok}[1]{\textcolor[rgb]{0.00,0.00,0.81}{#1}}
\newcommand{\DocumentationTok}[1]{\textcolor[rgb]{0.56,0.35,0.01}{\textbf{\textit{#1}}}}
\newcommand{\ErrorTok}[1]{\textcolor[rgb]{0.64,0.00,0.00}{\textbf{#1}}}
\newcommand{\ExtensionTok}[1]{#1}
\newcommand{\FloatTok}[1]{\textcolor[rgb]{0.00,0.00,0.81}{#1}}
\newcommand{\FunctionTok}[1]{\textcolor[rgb]{0.00,0.00,0.00}{#1}}
\newcommand{\ImportTok}[1]{#1}
\newcommand{\InformationTok}[1]{\textcolor[rgb]{0.56,0.35,0.01}{\textbf{\textit{#1}}}}
\newcommand{\KeywordTok}[1]{\textcolor[rgb]{0.13,0.29,0.53}{\textbf{#1}}}
\newcommand{\NormalTok}[1]{#1}
\newcommand{\OperatorTok}[1]{\textcolor[rgb]{0.81,0.36,0.00}{\textbf{#1}}}
\newcommand{\OtherTok}[1]{\textcolor[rgb]{0.56,0.35,0.01}{#1}}
\newcommand{\PreprocessorTok}[1]{\textcolor[rgb]{0.56,0.35,0.01}{\textit{#1}}}
\newcommand{\RegionMarkerTok}[1]{#1}
\newcommand{\SpecialCharTok}[1]{\textcolor[rgb]{0.00,0.00,0.00}{#1}}
\newcommand{\SpecialStringTok}[1]{\textcolor[rgb]{0.31,0.60,0.02}{#1}}
\newcommand{\StringTok}[1]{\textcolor[rgb]{0.31,0.60,0.02}{#1}}
\newcommand{\VariableTok}[1]{\textcolor[rgb]{0.00,0.00,0.00}{#1}}
\newcommand{\VerbatimStringTok}[1]{\textcolor[rgb]{0.31,0.60,0.02}{#1}}
\newcommand{\WarningTok}[1]{\textcolor[rgb]{0.56,0.35,0.01}{\textbf{\textit{#1}}}}
\usepackage{graphicx,grffile}
\makeatletter
\def\maxwidth{\ifdim\Gin@nat@width>\linewidth\linewidth\else\Gin@nat@width\fi}
\def\maxheight{\ifdim\Gin@nat@height>\textheight\textheight\else\Gin@nat@height\fi}
\makeatother
% Scale images if necessary, so that they will not overflow the page
% margins by default, and it is still possible to overwrite the defaults
% using explicit options in \includegraphics[width, height, ...]{}
\setkeys{Gin}{width=\maxwidth,height=\maxheight,keepaspectratio}
% Set default figure placement to htbp
\makeatletter
\def\fps@figure{htbp}
\makeatother
\setlength{\emergencystretch}{3em} % prevent overfull lines
\providecommand{\tightlist}{%
  \setlength{\itemsep}{0pt}\setlength{\parskip}{0pt}}
\setcounter{secnumdepth}{-\maxdimen} % remove section numbering

\title{Assignment 1 ETC1010 - 5510}
\usepackage{etoolbox}
\makeatletter
\providecommand{\subtitle}[1]{% add subtitle to \maketitle
  \apptocmd{\@title}{\par {\large #1 \par}}{}{}
}
\makeatother
\subtitle{New South Wales Crime Incidents Report}
\author{Jason Ching Yuen Siu}
\date{Thursday, March 25 2021}

\begin{document}
\maketitle

\hypertarget{instructions-to-students}{%
\section{Instructions to Students}\label{instructions-to-students}}

This assignment is designed to simulate a scenario in which you are
taking over someone's existing work and continuing with it to draw some
further insights.

This is a real world dataset taken from the New South Wales Bureau of
Crime Statistics and Research. The data can be found here at
\url{https://www.bocsar.nsw.gov.au/Documents/Datasets/SuburbData.zip}.
Specifically, the data file called ``SuburbData2019csv'' located in your
data folder inside the RStudio project will be used for this assignment.

You have just joined a consulting company as a data scientist. To give
you some experience and guidance, you are performing a quick summary of
the data while answering a number questions that the chief business
analytics leader has. This is not a formal report, but rather something
you are giving to your manager that describes the data with some
interesting insights.

Please make sure you read the hints throughout the assignment to help
guide you on the tasks.

The points allocated for each of the elements in the assignment are
marked next to the code for each question.

\hypertarget{marking-grades} of your total grade, and
  is marked out of 116 marks total. \textbf{Due on: Friday 26 March}.
\end{itemize}

For this assignment, you will need to upload the following into Moodle:

\begin{verbatim}
  - Your Rmd file,
  - The rendered html file, and
  - The PDF rendered file.
  
\end{verbatim}

\hypertarget{how-to-find-help-from-r-functions}{%
\subsection{How to find help from R
functions?}\label{how-to-find-help-from-r-functions}}

Remember, you can look up the help file for functions by typing:
\texttt{?function\_name}. For example, \texttt{?mean}. Feel free to
google questions you have about how to do other kinds of plots, and post
on the ``Assignment Discussion Forum'' any questions you have about the
assignment.

\hypertarget{how-to-complete-this-assignment}{%
\subsection{How to complete this
assignment?}\label{how-to-complete-this-assignment}}

To complete the assignment, you will need to fill in the blanks with
appropriate function names, arguments, or other names. These sections
are marked with \texttt{\_\_\_}. \textbf{At a minimum, your assignment
should be able to be ``knitted''} using the \texttt{Knit} button for
your Rmarkdown document.

If you want to look at what the assignment looks like in progress with
some of the R codes remaining invalid in the R code chunks, remember
that you can set the R chunk options to \texttt{eval\ =\ TRUE} like so:

\begin{Shaded}
\begin{Highlighting}[]
\BaseNTok{```\{r this-chunk-will-not-run, eval = TRUE\} `r''`}
\BaseNTok{ggplot()}
\BaseNTok{```}
\end{Highlighting}
\end{Shaded}

\textbf{If you use \texttt{eval\ =\ TRUE} or \texttt{cache\ =\ TRUE},
please remember to ensure that you have set to \texttt{eval\ =\ TRUE}
when you submit the assignment, to ensure all your R codes run.}

There are a few tricky bits that might require you to look back into
your previous R code chunks (that is intentionally done for you to
understand how things work within an Rmd file!)

You will be completing this assignment \textbf{INDIVIDUALLY}.

\hypertarget{due-date}{%
\subsection{Due Date}\label{due-date}}

This assignment is due in by close of business (5pm) on Friday, 26 March
2021. You will submit the assignment via Moodle. Please make sure you
add your name on the YAML part of this Rmd file.

\hypertarget{treatment}{%
\section{Treatment}\label{treatment}}

You work as a data scientist in the well-named consulting company,
``Consulting for You''.

It's your second day at the company, and you're taken to your desk. Your
boss says to you:

\begin{quote}
We have a data set with the crime statistics in New South Wales for the
past years!
\end{quote}

\begin{quote}
We've got a meeting coming up soon to get insights about the crime in
NSW. We want you to tell us about this data set and what we can do with
it.
\end{quote}

\begin{quote}
You're in with the new hires of data scientists here. We'd like you to
take a look at the data and tell me what the spreadsheet tells us. I've
written some questions on the report for you to answer.
\end{quote}

\begin{quote}
Most importantly, can you get this to me by \textbf{5pm, Friday, 26
March 2021}.
\end{quote}

\begin{quote}
Please read below and answer all the questions (ensure that you can knit
the file to produce an html file and a PDF file to hand them in to me
via Moodle):
\end{quote}

\hypertarget{load-all-the-libraries-that-you-need-here}{%
\section{Load all the libraries that you need
here}\label{load-all-the-libraries-that-you-need-here}}

\begin{Shaded}
\begin{Highlighting}[]
\KeywordTok{library}\NormalTok{(tidyverse)}
\end{Highlighting}
\end{Shaded}

\hypertarget{reading-and-preparing-data}{%
\section{Reading and preparing data}\label{reading-and-preparing-data}}

!change the working directory!

\begin{Shaded}
\begin{Highlighting}[]
\NormalTok{crime_dat <-}\StringTok{ }\KeywordTok{read_csv}\NormalTok{(}\StringTok{"data/SuburbData2019.csv"}\NormalTok{)}
\end{Highlighting}
\end{Shaded}

Note : Since from the question 20 onward need the suburban named
``Coogee'', therefore, I added a new ``Coogee''.

\begin{Shaded}
\begin{Highlighting}[]
\CommentTok{# I am selecting here only a portion of the data}
\CommentTok{# to reduce computation times.}

\NormalTok{crime_data <-crime_dat }\OperatorTok
\StringTok{  }\KeywordTok{select}\NormalTok{(}\OperatorTok{-}\KeywordTok{c}\NormalTok{(}\StringTok{`}\DataTypeTok{Jan 1995}\StringTok{`}\OperatorTok{:}\StringTok{`}\DataTypeTok{Jan 2010}\StringTok{`}\NormalTok{)) }\OperatorTok
\StringTok{  }\NormalTok{dplyr}\OperatorTok{::}\KeywordTok{filter}\NormalTok{(Suburb }\OperatorTok\StringTok{ }\KeywordTok{c}\NormalTok{(}\StringTok{"Chifley"}\NormalTok{,}
                              \StringTok{"Redfern"}\NormalTok{,}
                              \StringTok{"Clare"}\NormalTok{,}
                              \StringTok{"Paddington"}\NormalTok{,}
                              \StringTok{"Redfern"}\NormalTok{,}
                              \StringTok{"Zetland"}\NormalTok{,}
                              \StringTok{"Claymore"}\NormalTok{,}
                              \StringTok{"Congo"}\NormalTok{,}
                              \StringTok{"Yenda"}\NormalTok{, }
                              \StringTok{"Young"}\NormalTok{,}
                              \StringTok{"Yarra"}\NormalTok{,}
                              \StringTok{"Woodcroft"}\NormalTok{,}
                              \StringTok{"Woodhill"}\NormalTok{,}
                              \StringTok{"Warri"}\NormalTok{,}
                              \StringTok{"Waterloo"}\NormalTok{,}
                              \StringTok{"Randwick"}\NormalTok{,}
                              \StringTok{"Coogee"}\NormalTok{))}
\end{Highlighting}
\end{Shaded}

\hypertarget{question-1-display-the-first-10-rows-of-the-data-set}{%
\section{Question 1: Display the first 10 rows of the data
set}\label{question-1-display-the-first-10-rows-of-the-data-set}}

\textbf{Hint:} Check \emph{?head} in your R console

\begin{Shaded}
\begin{Highlighting}[]
\KeywordTok{head}\NormalTok{(crime_data, }\DecValTok{10}\NormalTok{)   }\CommentTok{# 1pt}
\end{Highlighting}
\end{Shaded}

\begin{verbatim}
## # A tibble: 10 x 122
##    Suburb `Offence catego~ Subcategory `Feb 2010` `Mar 2010` `Apr 2010`
##    <chr>  <chr>            <chr>            <dbl>      <dbl>      <dbl>
##  1 Chifl~ Homicide         Murder *             0          0          0
##  2 Chifl~ Homicide         Attempted ~          0          0          0
##  3 Chifl~ Homicide         Murder acc~          0          0          0
##  4 Chifl~ Homicide         Manslaught~          0          0          0
##  5 Chifl~ Assault          Domestic v~          1          0          1
##  6 Chifl~ Assault          Non-domest~          2          0          0
##  7 Chifl~ Assault          Assault Po~          0          0          0
##  8 Chifl~ Sexual offences  Sexual ass~          1          0          0
##  9 Chifl~ Sexual offences  Indecent a~          0          0          0
## 10 Chifl~ Abduction and k~ Abduction ~          0          0          0
## # ... with 116 more variables: `May 2010` <dbl>, `Jun 2010` <dbl>, `Jul
## #   2010` <dbl>, `Aug 2010` <dbl>, `Sep 2010` <dbl>, `Oct 2010` <dbl>, `Nov
## #   2010` <dbl>, `Dec 2010` <dbl>, `Jan 2011` <dbl>, `Feb 2011` <dbl>, `Mar
## #   2011` <dbl>, `Apr 2011` <dbl>, `May 2011` <dbl>, `Jun 2011` <dbl>, `Jul
## #   2011` <dbl>, `Aug 2011` <dbl>, `Sep 2011` <dbl>, `Oct 2011` <dbl>, `Nov
## #   2011` <dbl>, `Dec 2011` <dbl>, `Jan 2012` <dbl>, `Feb 2012` <dbl>, `Mar
## #   2012` <dbl>, `Apr 2012` <dbl>, `May 2012` <dbl>, `Jun 2012` <dbl>, `Jul
## #   2012` <dbl>, `Aug 2012` <dbl>, `Sep 2012` <dbl>, `Oct 2012` <dbl>, `Nov
## #   2012` <dbl>, `Dec 2012` <dbl>, `Jan 2013` <dbl>, `Feb 2013` <dbl>, `Mar
## #   2013` <dbl>, `Apr 2013` <dbl>, `May 2013` <dbl>, `Jun 2013` <dbl>, `Jul
## #   2013` <dbl>, `Aug 2013` <dbl>, `Sep 2013` <dbl>, `Oct 2013` <dbl>, `Nov
## #   2013` <dbl>, `Dec 2013` <dbl>, `Jan 2014` <dbl>, `Feb 2014` <dbl>, `Mar
## #   2014` <dbl>, `Apr 2014` <dbl>, `May 2014` <dbl>, `Jun 2014` <dbl>, `Jul
## #   2014` <dbl>, `Aug 2014` <dbl>, `Sep 2014` <dbl>, `Oct 2014` <dbl>, `Nov
## #   2014` <dbl>, `Dec 2014` <dbl>, `Jan 2015` <dbl>, `Feb 2015` <dbl>, `Mar
## #   2015` <dbl>, `Apr 2015` <dbl>, `May 2015` <dbl>, `Jun 2015` <dbl>, `Jul
## #   2015` <dbl>, `Aug 2015` <dbl>, `Sep 2015` <dbl>, `Oct 2015` <dbl>, `Nov
## #   2015` <dbl>, `Dec 2015` <dbl>, `Jan 2016` <dbl>, `Feb 2016` <dbl>, `Mar
## #   2016` <dbl>, `Apr 2016` <dbl>, `May 2016` <dbl>, `Jun 2016` <dbl>, `Jul
## #   2016` <dbl>, `Aug 2016` <dbl>, `Sep 2016` <dbl>, `Oct 2016` <dbl>, `Nov
## #   2016` <dbl>, `Dec 2016` <dbl>, `Jan 2017` <dbl>, `Feb 2017` <dbl>, `Mar
## #   2017` <dbl>, `Apr 2017` <dbl>, `May 2017` <dbl>, `Jun 2017` <dbl>, `Jul
## #   2017` <dbl>, `Aug 2017` <dbl>, `Sep 2017` <dbl>, `Oct 2017` <dbl>, `Nov
## #   2017` <dbl>, `Dec 2017` <dbl>, `Jan 2018` <dbl>, `Feb 2018` <dbl>, `Mar
## #   2018` <dbl>, `Apr 2018` <dbl>, `May 2018` <dbl>, `Jun 2018` <dbl>, `Jul
## #   2018` <dbl>, `Aug 2018` <dbl>, ...
\end{verbatim}

\hypertarget{question-2-how-many-variables-and-observations-do-we-have}{%
\section{Question 2: How many variables and observations do we
have?}\label{question-2-how-many-variables-and-observations-do-we-have}}

\textbf{Hint:} Look for help \emph{?dim} in your R console and remember
that variables are in columns and observations in rows. \emph{dim()}
returns the number of rows and the number of columns in the data set (in
that order)

\begin{verbatim}
## [1] 992 122
\end{verbatim}

The number of variables are 992 (1pt) and the number of rows are 122
(1pt)

\hypertarget{question-3-what-are-the-names-of-the-first-20-vaiables-in-this-data-set}{%
\section{Question 3: What are the names of the first 20 vaiables in this
data
set?}\label{question-3-what-are-the-names-of-the-first-20-vaiables-in-this-data-set}}

\begin{Shaded}
\begin{Highlighting}[]
\KeywordTok{names}\NormalTok{(crime_data)[}\DecValTok{1}\OperatorTok{:}\DecValTok{20}\NormalTok{]   }\CommentTok{#1pt}
\end{Highlighting}
\end{Shaded}

\begin{verbatim}
##  [1] "Suburb"           "Offence category" "Subcategory"      "Feb 2010"        
##  [5] "Mar 2010"         "Apr 2010"         "May 2010"         "Jun 2010"        
##  [9] "Jul 2010"         "Aug 2010"         "Sep 2010"         "Oct 2010"        
## [13] "Nov 2010"         "Dec 2010"         "Jan 2011"         "Feb 2011"        
## [17] "Mar 2011"         "Apr 2011"         "May 2011"         "Jun 2011"
\end{verbatim}

\hypertarget{question-4-rename-the-variable-of-offence-category-to-offence_category-and-show-the-names-of-the-first-4-variables-in-the-data-set}{%
\section{Question 4: Rename the variable of ``Offence category'' to
``Offence\_category'' and show the names of the first 4 variables in the
data
set}\label{question-4-rename-the-variable-of-offence-category-to-offence_category-and-show-the-names-of-the-first-4-variables-in-the-data-set}}

\begin{Shaded}
\begin{Highlighting}[]
\NormalTok{crime <-}\StringTok{ }\NormalTok{crime_data }\OperatorTok
\StringTok{  }\KeywordTok{rename}\NormalTok{(}\StringTok{`}\DataTypeTok{Offence_category}\StringTok{`}\NormalTok{ =}\StringTok{ `}\DataTypeTok{Offence category}\StringTok{`}\NormalTok{)  }\CommentTok{# 1pt}

\KeywordTok{names}\NormalTok{(crime)[}\DecValTok{1}\OperatorTok{:}\DecValTok{4}\NormalTok{] }\CommentTok{#1pt}
\end{Highlighting}
\end{Shaded}

\begin{verbatim}
## [1] "Suburb"           "Offence_category" "Subcategory"      "Feb 2010"
\end{verbatim}

\hypertarget{question-5-change-the-crime-data-suburbdata2019csv-into-long-format-so-that-all-the-years-are-grouped-together-into-a-variable-called-year-and-the-corresponding-incidents-count-into-a-variable-called-incidents}{%
\section{Question 5: Change the ``crime'' data (``SuburbData2019csv'')
into long format so that all the years are grouped together into a
variable called ``year'' and the corresponding incidents count into a
variable called
``incidents''}\label{question-5-change-the-crime-data-suburbdata2019csv-into-long-format-so-that-all-the-years-are-grouped-together-into-a-variable-called-year-and-the-corresponding-incidents-count-into-a-variable-called-incidents}}

\begin{Shaded}
\begin{Highlighting}[]
\NormalTok{crime_long <-}\StringTok{ }\NormalTok{crime }\OperatorTok
\StringTok{  }\KeywordTok{pivot_longer}\NormalTok{(}\DataTypeTok{cols =} \StringTok{"Feb 2010"}\OperatorTok{:}\StringTok{"Dec 2019"}\NormalTok{,  }\CommentTok{# 2pt}
               \DataTypeTok{names_to =} \StringTok{"year"}\NormalTok{,             }\CommentTok{# 1pt}
               \DataTypeTok{values_to =} \StringTok{"incidents"}\NormalTok{)       }\CommentTok{# 1pt}

\KeywordTok{head}\NormalTok{(crime_long)     }\CommentTok{# 1pt}
\end{Highlighting}
\end{Shaded}

\begin{verbatim}
## # A tibble: 6 x 5
##   Suburb  Offence_category Subcategory year     incidents
##   <chr>   <chr>            <chr>       <chr>        <dbl>
## 1 Chifley Homicide         Murder *    Feb 2010         0
## 2 Chifley Homicide         Murder *    Mar 2010         0
## 3 Chifley Homicide         Murder *    Apr 2010         0
## 4 Chifley Homicide         Murder *    May 2010         0
## 5 Chifley Homicide         Murder *    Jun 2010         0
## 6 Chifley Homicide         Murder *    Jul 2010         0
\end{verbatim}

\hypertarget{question-6-separate-the-column-year-into-two-columns-with-names-month-and-year.-display-the-first-3-lines-of-the-data-set-to-show-the-updated-data-set}{%
\section{Question 6: Separate the column ``year'' into two columns with
names ``Month'' and ``Year''. Display the first 3 lines of the data set
to show the updated data
set}\label{question-6-separate-the-column-year-into-two-columns-with-names-month-and-year.-display-the-first-3-lines-of-the-data-set-to-show-the-updated-data-set}}

\begin{Shaded}
\begin{Highlighting}[]
\NormalTok{crime_long_new <-}\StringTok{ }\NormalTok{crime_long }\OperatorTok
\StringTok{  }\KeywordTok{separate}\NormalTok{(}\DataTypeTok{col =}\NormalTok{ year,                        }\CommentTok{# 1pt}
           \DataTypeTok{into =} \KeywordTok{c}\NormalTok{(}\StringTok{"Month"}\NormalTok{, }\StringTok{"Year"}\NormalTok{), }\StringTok{" "}\NormalTok{ )   }\CommentTok{# 2pt}

  \KeywordTok{head}\NormalTok{(crime_long_new,}\DecValTok{3}\NormalTok{)                        }\CommentTok{# 1pt}
\end{Highlighting}
\end{Shaded}

\begin{verbatim}
## # A tibble: 3 x 6
##   Suburb  Offence_category Subcategory Month Year  incidents
##   <chr>   <chr>            <chr>       <chr> <chr>     <dbl>
## 1 Chifley Homicide         Murder *    Feb   2010          0
## 2 Chifley Homicide         Murder *    Mar   2010          0
## 3 Chifley Homicide         Murder *    Apr   2010          0
\end{verbatim}

\hypertarget{question-7-if-you-look-at-the-data-crime_long_new-you-will-notice-that-the-variable-of-year-is-coded-as-character.-in-this-section-we-are-going-to-convert-the-variable-of-year-to-a-numeric-variable}{%
\section{\texorpdfstring{Question 7: If you look at the data
\emph{crime\_long\_new}, you will notice that the variable of ``Year''
is coded as character. In this section, we are going to convert the
variable of ``Year'' to a numeric
variable}{Question 7: If you look at the data crime\_long\_new, you will notice that the variable of ``Year'' is coded as character. In this section, we are going to convert the variable of ``Year'' to a numeric variable}}\label{question-7-if-you-look-at-the-data-crime_long_new-you-will-notice-that-the-variable-of-year-is-coded-as-character.-in-this-section-we-are-going-to-convert-the-variable-of-year-to-a-numeric-variable}}

\begin{Shaded}
\begin{Highlighting}[]
\NormalTok{crime_long_new }\OperatorTok
\StringTok{  }\KeywordTok{mutate}\NormalTok{(}\DataTypeTok{Year =} \KeywordTok{as.numeric}\NormalTok{(Year))   }\CommentTok{# 1pt}
\end{Highlighting}
\end{Shaded}

\begin{verbatim}
## # A tibble: 118,048 x 6
##    Suburb  Offence_category Subcategory Month  Year incidents
##    <chr>   <chr>            <chr>       <chr> <dbl>     <dbl>
##  1 Chifley Homicide         Murder *    Feb    2010         0
##  2 Chifley Homicide         Murder *    Mar    2010         0
##  3 Chifley Homicide         Murder *    Apr    2010         0
##  4 Chifley Homicide         Murder *    May    2010         0
##  5 Chifley Homicide         Murder *    Jun    2010         0
##  6 Chifley Homicide         Murder *    Jul    2010         0
##  7 Chifley Homicide         Murder *    Aug    2010         0
##  8 Chifley Homicide         Murder *    Sep    2010         0
##  9 Chifley Homicide         Murder *    Oct    2010         0
## 10 Chifley Homicide         Murder *    Nov    2010         0
## # ... with 118,038 more rows
\end{verbatim}

\begin{Shaded}
\begin{Highlighting}[]
\KeywordTok{head}\NormalTok{(crime_long_new }\OperatorTok
\StringTok{  }\KeywordTok{mutate}\NormalTok{(}\DataTypeTok{Year =} \KeywordTok{as.numeric}\NormalTok{(Year)) )                }\CommentTok{# 1pt}
\end{Highlighting}
\end{Shaded}

\begin{verbatim}
## # A tibble: 6 x 6
##   Suburb  Offence_category Subcategory Month  Year incidents
##   <chr>   <chr>            <chr>       <chr> <dbl>     <dbl>
## 1 Chifley Homicide         Murder *    Feb    2010         0
## 2 Chifley Homicide         Murder *    Mar    2010         0
## 3 Chifley Homicide         Murder *    Apr    2010         0
## 4 Chifley Homicide         Murder *    May    2010         0
## 5 Chifley Homicide         Murder *    Jun    2010         0
## 6 Chifley Homicide         Murder *    Jul    2010         0
\end{verbatim}

\hypertarget{question-8-display-the-years-in-the-data-set.-how-many-years-are-included-in-this-data-set}{%
\section{Question 8: Display the years in the data set. How many years
are included in this data
set?}\label{question-8-display-the-years-in-the-data-set.-how-many-years-are-included-in-this-data-set}}

\begin{quote}
Remember that you can learn more about what these functions by typing:
\texttt{?unique} or \texttt{?length} into the R console.
\end{quote}

\begin{Shaded}
\begin{Highlighting}[]
\KeywordTok{unique}\NormalTok{(crime_long_new}\OperatorTok{$}\NormalTok{Year)    }\CommentTok{# 1pt}
\end{Highlighting}
\end{Shaded}

\begin{verbatim}
##  [1] "2010" "2011" "2012" "2013" "2014" "2015" "2016" "2017" "2018" "2019"
\end{verbatim}

\begin{Shaded}
\begin{Highlighting}[]
\CommentTok{# length tell us the length or longitude of a variable or a vector}
\KeywordTok{length}\NormalTok{(}\KeywordTok{unique}\NormalTok{(crime_long_new}\OperatorTok{$}\NormalTok{Year))   }\CommentTok{#1pt}
\end{Highlighting}
\end{Shaded}

\begin{verbatim}
## [1] 10
\end{verbatim}

\hypertarget{question-9-how-many-different-suburbs-are-there-in-the-data-set}{%
\section{Question 9: How many different suburbs are there in the data
set?}\label{question-9-how-many-different-suburbs-are-there-in-the-data-set}}

\begin{Shaded}
\begin{Highlighting}[]
\KeywordTok{length}\NormalTok{(}\KeywordTok{unique}\NormalTok{(crime_long_new}\OperatorTok{$}\NormalTok{Suburb))     }\CommentTok{# 1pt}
\end{Highlighting}
\end{Shaded}

\begin{verbatim}
## [1] 16
\end{verbatim}

\begin{Shaded}
\begin{Highlighting}[]
\KeywordTok{n_distinct}\NormalTok{(crime_long_new}\OperatorTok{$}\NormalTok{Suburb) }\CommentTok{# 1pt }
\end{Highlighting}
\end{Shaded}

\begin{verbatim}
## [1] 16
\end{verbatim}

\hypertarget{question-10-how-many-incidents-do-we-have-per-offence_category-in-total-for-2019}{%
\section{Question 10: How many incidents do we have per
``Offence\_category'' in total for
2019?}\label{question-10-how-many-incidents-do-we-have-per-offence_category-in-total-for-2019}}

\begin{Shaded}
\begin{Highlighting}[]
\NormalTok{crime_long_new }\OperatorTok\StringTok{ }
\StringTok{  }\NormalTok{dplyr}\OperatorTok{::}\KeywordTok{filter}\NormalTok{(Year }\OperatorTok{==}\StringTok{ "2019"}\NormalTok{) }\OperatorTok\StringTok{    }\CommentTok{# 1pt}
\StringTok{  }\KeywordTok{count}\NormalTok{(Offence_category , }\DataTypeTok{wt =}\NormalTok{ incidents)              }\CommentTok{# 1pt}
\end{Highlighting}
\end{Shaded}

\begin{verbatim}
## # A tibble: 21 x 2
##    Offence_category                          n
##    <chr>                                 <dbl>
##  1 Abduction and kidnapping                  1
##  2 Against justice procedures             1950
##  3 Arson                                    60
##  4 Assault                                1396
##  5 Betting and gaming offences               1
##  6 Blackmail and extortion                   2
##  7 Disorderly conduct                      429
##  8 Drug offences                          1416
##  9 Homicide                                  2
## 10 Intimidation, stalking and harassment   566
## # ... with 11 more rows
\end{verbatim}

\hypertarget{question-11-which-is-the-offence_category-with-highest-number-of-incidents-in-2019}{%
\section{Question 11: Which is the ``Offence\_category'' with highest
number of incidents in
2019?}\label{question-11-which-is-the-offence_category-with-highest-number-of-incidents-in-2019}}

\begin{Shaded}
\begin{Highlighting}[]
\NormalTok{crime_long_new }\OperatorTok\StringTok{ }
\StringTok{  }\NormalTok{dplyr}\OperatorTok{::}\KeywordTok{filter}\NormalTok{(Year }\OperatorTok{==}\StringTok{ "2019"}\NormalTok{) }\OperatorTok\StringTok{    }\CommentTok{#  1pt}
\StringTok{  }\KeywordTok{count}\NormalTok{(Offence_category , }\DataTypeTok{wt =}\NormalTok{ incidents, }\DataTypeTok{sort =} \OtherTok{TRUE}\NormalTok{) }\CommentTok{# 1pt}
\end{Highlighting}
\end{Shaded}

\begin{verbatim}
## # A tibble: 21 x 2
##    Offence_category                          n
##    <chr>                                 <dbl>
##  1 Theft                                  4061
##  2 Against justice procedures             1950
##  3 Drug offences                          1416
##  4 Assault                                1396
##  5 Malicious damage to property           1093
##  6 Intimidation, stalking and harassment   566
##  7 Transport regulatory offences           517
##  8 Disorderly conduct                      429
##  9 Liquor offences                         356
## 10 Sexual offences                         273
## # ... with 11 more rows
\end{verbatim}

\hypertarget{question-12-how-many-offences-are-there-in-each-subcategory-of-the-offence_category-of-homicide}{%
\section{\texorpdfstring{Question 12: How many offences are there in
each Subcategory of the ``Offence\_category'' of
\emph{Homicide}?}{Question 12: How many offences are there in each Subcategory of the ``Offence\_category'' of Homicide?}}\label{question-12-how-many-offences-are-there-in-each-subcategory-of-the-offence_category-of-homicide}}

\begin{Shaded}
\begin{Highlighting}[]
\NormalTok{crime_long_new }\OperatorTok\StringTok{ }
\StringTok{  }\NormalTok{dplyr}\OperatorTok{::}\KeywordTok{filter}\NormalTok{(Offence_category }\OperatorTok{==}\StringTok{ "Homicide"}\NormalTok{) }\OperatorTok\StringTok{    }\CommentTok{# 1pt}
\StringTok{  }\KeywordTok{group_by}\NormalTok{(Subcategory) }\OperatorTok\StringTok{                            }\CommentTok{# 1pt}
\StringTok{  }\KeywordTok{summarise}\NormalTok{(}\DataTypeTok{Number_of_incidents =} \KeywordTok{sum}\NormalTok{(incidents))      }\CommentTok{# 1pt}
\end{Highlighting}
\end{Shaded}

\begin{verbatim}
## # A tibble: 4 x 2
##   Subcategory                  Number_of_incidents
##   <chr>                                      <dbl>
## 1 Attempted murder                               3
## 2 Manslaughter *                                 1
## 3 Murder *                                      14
## 4 Murder accessory, conspiracy                   1
\end{verbatim}

\hypertarget{question-13-select-the-suburb-called-paddington-and-calculate-the-number-of-incidents-for-offence_category-of-drug-offences-then-calculate-the-total-number-of-incidents-for-each-subcategory.-finally-show-a-table-arranged-by-number_of_-incidents-high-to-low}{%
\section{Question 13: Select the suburb called ``Paddington'' and
calculate the number of incidents for ``Offence\_category'' of ``Drug
offences'' then calculate the total number of incidents for each
Subcategory. Finally, show a table arranged by ``Number\_of\_
incidents'' (high to
low)}\label{question-13-select-the-suburb-called-paddington-and-calculate-the-number-of-incidents-for-offence_category-of-drug-offences-then-calculate-the-total-number-of-incidents-for-each-subcategory.-finally-show-a-table-arranged-by-number_of_-incidents-high-to-low}}

\begin{Shaded}
\begin{Highlighting}[]
\NormalTok{Paddington <-}\StringTok{ }\NormalTok{crime_long_new }\OperatorTok\StringTok{ }
\StringTok{  }\NormalTok{dplyr}\OperatorTok{::}\KeywordTok{filter}\NormalTok{(Suburb}\OperatorTok{==}\StringTok{ "Paddington"}\NormalTok{,                       }\CommentTok{# 2pt}
\NormalTok{                 Offence_category }\OperatorTok{==}\StringTok{ "Drug offences"}\NormalTok{) }\OperatorTok\StringTok{      }\CommentTok{#1pt}
\StringTok{          }\KeywordTok{group_by}\NormalTok{(Subcategory) }\OperatorTok\StringTok{                            }\CommentTok{# 1pt}
\StringTok{          }\KeywordTok{summarise}\NormalTok{(}\DataTypeTok{Number_of_incidents =} \KeywordTok{sum}\NormalTok{(incidents) )  }\OperatorTok\StringTok{    }\CommentTok{# 1pt}
\StringTok{          }\KeywordTok{arrange}\NormalTok{(}\KeywordTok{desc}\NormalTok{(Number_of_incidents))    }\CommentTok{# 1pt}
 
\KeywordTok{head}\NormalTok{(Paddington)                             }\CommentTok{# 1pt}
\end{Highlighting}
\end{Shaded}

\begin{verbatim}
## # A tibble: 6 x 2
##   Subcategory                           Number_of_incidents
##   <chr>                                               <dbl>
## 1 Possession and/or use of cannabis                     154
## 2 Possession and/or use of cocaine                      111
## 3 Possession and/or use of other drugs                   82
## 4 Other drug offences                                    73
## 5 Dealing, trafficking in cocaine                        68
## 6 Possession and/or use of amphetamines                  57
\end{verbatim}

\hypertarget{question-14-lets-have-a-look-at-the-changes-over-time-for-possession-andor-use-of-cannabis-in-the-suburb-of-paddington}{%
\section{Question 14: Let's have a look at the changes over time for
``Possession and/or use of cannabis'' in the suburb of
Paddington}\label{question-14-lets-have-a-look-at-the-changes-over-time-for-possession-andor-use-of-cannabis-in-the-suburb-of-paddington}}

To answer this question, we need to first filter the ``Suburb'' and the
``Subcategory''. Then, group incident by year and finally sum the number
of incidents for each year

\begin{Shaded}
\begin{Highlighting}[]
\NormalTok{Paddington_cannabis <-}\StringTok{ }\NormalTok{crime_long_new }\OperatorTok\StringTok{ }
\StringTok{  }\NormalTok{dplyr}\OperatorTok{::}\KeywordTok{filter}\NormalTok{( Suburb }\OperatorTok{==}\StringTok{ "Paddington"}\NormalTok{,               }\CommentTok{# 1pt}
\NormalTok{                 Subcategory }\OperatorTok{==}\StringTok{ "Possession and/or use of cannabis"}\NormalTok{) }\OperatorTok\StringTok{   }\CommentTok{# 1pt}
\StringTok{          }\KeywordTok{group_by}\NormalTok{(Year) }\OperatorTok\StringTok{                                               }\CommentTok{# 1pt}
\StringTok{          }\KeywordTok{summarise}\NormalTok{(}\DataTypeTok{Number_of_incidents =} \KeywordTok{sum}\NormalTok{( incidents) )}\OperatorTok\StringTok{              }\CommentTok{#1pt}
\StringTok{  }\KeywordTok{mutate}\NormalTok{(}\DataTypeTok{Year =} \KeywordTok{as.numeric}\NormalTok{(Year))    }\CommentTok{# 1pt}
 

\KeywordTok{head}\NormalTok{(Paddington_cannabis,}\DecValTok{3}\NormalTok{)  }\CommentTok{# 1pt}
\end{Highlighting}
\end{Shaded}

\begin{verbatim}
## # A tibble: 3 x 2
##    Year Number_of_incidents
##   <dbl>               <dbl>
## 1  2010                  17
## 2  2011                  17
## 3  2012                  15
\end{verbatim}

\hypertarget{question-15-create-a-line-plot-to-display-the-trend-of-the-incidents-that-you-calculated-for-paddington}{%
\section{Question 15: Create a line plot to display the trend of the
incidents that you calculated for
Paddington}\label{question-15-create-a-line-plot-to-display-the-trend-of-the-incidents-that-you-calculated-for-paddington}}

On the x-axis you should have ``Year'' and on the y-axis you should
display ``Number\_of\_incidents''

\begin{Shaded}
\begin{Highlighting}[]
\KeywordTok{ggplot}\NormalTok{(Paddington_cannabis, }\KeywordTok{aes}\NormalTok{( }\DataTypeTok{x =}\NormalTok{ Year, }\DataTypeTok{y =}\NormalTok{ Number_of_incidents)) }\OperatorTok{+}\StringTok{ }\CommentTok{# 2pt }
\StringTok{  }\KeywordTok{geom_line}\NormalTok{()                     }\CommentTok{# 1pt}
\end{Highlighting}
\end{Shaded}

\includegraphics{Assignment_1_ETC1010_5510_files/figure-latex/unnamed-chunk-8-1.pdf}

\hypertarget{question-16-create-the-same-plot-as-in-question-15-but-now-include-also-the-suburb-called-randwick-you-will-see-two-trends-in-the-same-plot.-make-sure-that-the-variable-of-suburb-is-defined-as-a-factor}{%
\section{\texorpdfstring{Question 16: Create the same plot as in
Question 15 but now include also the suburb called ``Randwick'' (you
will see two trends in the same plot). Make sure that the variable of
``Suburb'' is defined as a
\emph{factor}}{Question 16: Create the same plot as in Question 15 but now include also the suburb called ``Randwick'' (you will see two trends in the same plot). Make sure that the variable of ``Suburb'' is defined as a factor}}\label{question-16-create-the-same-plot-as-in-question-15-but-now-include-also-the-suburb-called-randwick-you-will-see-two-trends-in-the-same-plot.-make-sure-that-the-variable-of-suburb-is-defined-as-a-factor}}

\begin{Shaded}
\begin{Highlighting}[]
\NormalTok{both_cannabis <-}\StringTok{ }\NormalTok{crime_long_new }\OperatorTok\StringTok{ }
\StringTok{  }\NormalTok{dplyr}\OperatorTok{::}\KeywordTok{filter}\NormalTok{( Suburb }\OperatorTok\StringTok{ }\KeywordTok{c}\NormalTok{(}\StringTok{"Randwick"}\NormalTok{, }\StringTok{"Paddington"}\NormalTok{),          }\CommentTok{# 1pt }
\NormalTok{                 Subcategory }\OperatorTok{==}\StringTok{ "Possession and/or use of cannabis"}\NormalTok{) }\OperatorTok\StringTok{   }\CommentTok{# 1pt}
\StringTok{          }\KeywordTok{group_by}\NormalTok{(Year, Suburb) }\OperatorTok\StringTok{                              }\CommentTok{# 1pt}
\StringTok{          }\KeywordTok{summarise}\NormalTok{(}\DataTypeTok{Number_of_incidents =} \KeywordTok{sum}\NormalTok{( incidents))  }\OperatorTok\StringTok{     }\CommentTok{# 1pt}
\StringTok{  }\KeywordTok{mutate}\NormalTok{(}
         \DataTypeTok{Year =} \KeywordTok{as.numeric}\NormalTok{(Year),                                 }\CommentTok{# 1pt}
         \DataTypeTok{Suburb =} \KeywordTok{as.factor}\NormalTok{(Suburb)}
\NormalTok{         )                              }\CommentTok{# 1pt}
  

\KeywordTok{ggplot}\NormalTok{(both_cannabis, }\KeywordTok{aes}\NormalTok{( }\DataTypeTok{x =}\NormalTok{ Year,      }\CommentTok{# 1pt}
                           \DataTypeTok{y =}\NormalTok{ Number_of_incidents, }\CommentTok{# 1pt}
                           \DataTypeTok{color =}\NormalTok{ Suburb)) }\OperatorTok{+}\StringTok{   }\CommentTok{# 1pt}
\StringTok{  }\KeywordTok{geom_line}\NormalTok{()   }\CommentTok{# 1pt}
\end{Highlighting}
\end{Shaded}

\includegraphics{Assignment_1_ETC1010_5510_files/figure-latex/unnamed-chunk-9-1.pdf}

\hypertarget{question-17-lets-now-look-at-the-total-number-of-crime-incidents-in-nsw-and-create-a-plot-to-visualize-the-trend}{%
\section{Question 17: Let's now look at the total number of crime
incidents in NSW and create a plot to visualize the
trend}\label{question-17-lets-now-look-at-the-total-number-of-crime-incidents-in-nsw-and-create-a-plot-to-visualize-the-trend}}

\begin{Shaded}
\begin{Highlighting}[]
\NormalTok{crime_long_new }\OperatorTok\StringTok{ }
\StringTok{  }\NormalTok{dplyr}\OperatorTok{::}\KeywordTok{select}\NormalTok{( Year,   }\CommentTok{# 1pt}
\NormalTok{                 incidents) }\OperatorTok\StringTok{  }\CommentTok{# 1pt}
\StringTok{          }\KeywordTok{group_by}\NormalTok{(Year) }\OperatorTok\StringTok{    }\CommentTok{# 1pt}
\StringTok{          }\KeywordTok{summarise}\NormalTok{(}\DataTypeTok{Number_of_incidents =} \KeywordTok{sum}\NormalTok{(incidents)) }\OperatorTok\StringTok{  }\CommentTok{# 1pt}
\StringTok{  }\KeywordTok{mutate}\NormalTok{(}\DataTypeTok{Year =} \KeywordTok{as.numeric}\NormalTok{(Year)) }\OperatorTok\StringTok{    }\CommentTok{# 1pt}
\StringTok{  }\KeywordTok{ggplot}\NormalTok{(}\KeywordTok{aes}\NormalTok{(}\DataTypeTok{x =}\NormalTok{ Year, }\DataTypeTok{y =}\NormalTok{ Number_of_incidents )) }\OperatorTok{+}\StringTok{   }\CommentTok{# 1pt}
\StringTok{  }\KeywordTok{geom_line}\NormalTok{()  }\CommentTok{# 1pt}
\end{Highlighting}
\end{Shaded}

\includegraphics{Assignment_1_ETC1010_5510_files/figure-latex/unnamed-chunk-10-1.pdf}

\hypertarget{question-18-now-lets-change-the-background-color-of-the-plot-to-white-using-the-theme_bw}{%
\section{\texorpdfstring{Question 18: Now, let's change the background
color of the plot to white using the
\emph{theme\_bw()}}{Question 18: Now, let's change the background color of the plot to white using the theme\_bw()}}\label{question-18-now-lets-change-the-background-color-of-the-plot-to-white-using-the-theme_bw}}

\begin{Shaded}
\begin{Highlighting}[]
\CommentTok{## !Make sure you doublr check the question!}
\NormalTok{crime_long_new }\OperatorTok\StringTok{ }
\StringTok{  }\NormalTok{dplyr}\OperatorTok{::}\KeywordTok{select}\NormalTok{( Year,   }\CommentTok{# 1pt}
\NormalTok{                 incidents) }\OperatorTok\StringTok{  }\CommentTok{# 1pt}
\StringTok{          }\KeywordTok{group_by}\NormalTok{(Year) }\OperatorTok\StringTok{    }\CommentTok{# 1pt}
\StringTok{          }\KeywordTok{summarise}\NormalTok{(}\DataTypeTok{Number_of_incidents =} \KeywordTok{sum}\NormalTok{(incidents)) }\OperatorTok\StringTok{  }\CommentTok{# 1pt}
\StringTok{  }\KeywordTok{mutate}\NormalTok{(}\DataTypeTok{Year =} \KeywordTok{as.numeric}\NormalTok{(Year)) }\OperatorTok\StringTok{    }\CommentTok{# 1pt}
\StringTok{  }\KeywordTok{ggplot}\NormalTok{(}\KeywordTok{aes}\NormalTok{(}\DataTypeTok{x =}\NormalTok{ Year, }\DataTypeTok{y =}\NormalTok{ Number_of_incidents )) }\OperatorTok{+}\StringTok{   }\CommentTok{# 1pt}
\StringTok{  }\KeywordTok{geom_line}\NormalTok{() }\OperatorTok{+}\StringTok{ }\CommentTok{# 1pt}
\StringTok{  }\KeywordTok{theme_bw}\NormalTok{()  }\CommentTok{# 1pt}
\end{Highlighting}
\end{Shaded}

\includegraphics{Assignment_1_ETC1010_5510_files/figure-latex/unnamed-chunk-11-1.pdf}

\hypertarget{question-19-lets-change-the-line-color-to-green-and-replace-it-with-a-dotted-line}{%
\section{Question 19: Let's change the line color to green and replace
it with a dotted
line}\label{question-19-lets-change-the-line-color-to-green-and-replace-it-with-a-dotted-line}}

\begin{Shaded}
\begin{Highlighting}[]
\NormalTok{crime_long_new }\OperatorTok\StringTok{ }
\StringTok{  }\NormalTok{dplyr}\OperatorTok{::}\KeywordTok{select}\NormalTok{(incidents,  }\CommentTok{# 1pt}
\NormalTok{                 Year) }\OperatorTok\StringTok{  }\CommentTok{# 1pt}
\StringTok{          }\KeywordTok{group_by}\NormalTok{(Year) }\OperatorTok\StringTok{  }\CommentTok{# 1pt}
\StringTok{         }\KeywordTok{summarise}\NormalTok{(}\DataTypeTok{Number_of_incidents =} \KeywordTok{sum}\NormalTok{(incidents)) }\OperatorTok\StringTok{  }\CommentTok{# 1pt}
\StringTok{  }\KeywordTok{mutate}\NormalTok{(}\DataTypeTok{Year =} \KeywordTok{as.numeric}\NormalTok{(Year)) }\OperatorTok\StringTok{  }\CommentTok{# 1pt}
\StringTok{  }\KeywordTok{ggplot}\NormalTok{(}\KeywordTok{aes}\NormalTok{(}\DataTypeTok{x =}\NormalTok{ Year, }\DataTypeTok{y =}\NormalTok{ Number_of_incidents )) }\OperatorTok{+}\StringTok{  }\CommentTok{# 1pt}
\StringTok{  }\KeywordTok{geom_line}\NormalTok{(}\DataTypeTok{linetype =} \StringTok{"dotted"}\NormalTok{, }\DataTypeTok{color =}\StringTok{"Green"}\NormalTok{)  }\CommentTok{# 1pt}
\end{Highlighting}
\end{Shaded}

\includegraphics{Assignment_1_ETC1010_5510_files/figure-latex/unnamed-chunk-12-1.pdf}

\hypertarget{question-20-now-lets-look-at-the-total-number-of-crime-incidents-for-the-suburbs-of-redfern-coogee-and-zetland-by-creating-a-bar-plot-where-we-have-the-incidents-per-suburb-by-year-next-to-each-other}{%
\section{Question 20: Now, let's look at the total number of crime
incidents for the suburbs of Redfern, Coogee, and Zetland by creating a
bar plot where we have the incidents per suburb by year next to each
other}\label{question-20-now-lets-look-at-the-total-number-of-crime-incidents-for-the-suburbs-of-redfern-coogee-and-zetland-by-creating-a-bar-plot-where-we-have-the-incidents-per-suburb-by-year-next-to-each-other}}

\begin{Shaded}
\begin{Highlighting}[]
\CommentTok{##!missing!}
\NormalTok{comparison_data<-}\StringTok{ }\NormalTok{crime_long_new }\OperatorTok
\StringTok{  }\NormalTok{dplyr}\OperatorTok{::}\KeywordTok{select}\NormalTok{(Suburb ,  }\CommentTok{# 1pt}
\NormalTok{                Year,    }\CommentTok{# 1pt }
\NormalTok{                incidents) }\OperatorTok\StringTok{  }\CommentTok{# 1pt}
\StringTok{  }\NormalTok{dplyr}\OperatorTok{::}\KeywordTok{filter}\NormalTok{( Suburb  }\OperatorTok\StringTok{ }\KeywordTok{c}\NormalTok{(}\StringTok{"Redfern"}\NormalTok{, }\StringTok{"Coogee"}\NormalTok{, }\StringTok{"Zetland"}\NormalTok{)) }\OperatorTok\StringTok{  }\CommentTok{# 1pt}
\StringTok{          }\KeywordTok{group_by}\NormalTok{( Suburb,Year) }\OperatorTok\StringTok{ }\CommentTok{# 1pt}
\StringTok{          }\KeywordTok{summarise}\NormalTok{(}\DataTypeTok{Number_of_incidents =} \KeywordTok{sum}\NormalTok{(incidents)) }\CommentTok{# 1pt}

\CommentTok{## for the sake of the question 23, Year is changed to numeric}
\NormalTok{  comparison_data <-}\StringTok{ }\NormalTok{comparison_data }\OperatorTok\StringTok{ }\KeywordTok{mutate}\NormalTok{(}\DataTypeTok{Year =} \KeywordTok{as.numeric}\NormalTok{(Year))}
  


  \KeywordTok{ggplot}\NormalTok{(comparison_data, }\KeywordTok{aes}\NormalTok{(}\DataTypeTok{x =}\NormalTok{ Year,    }\CommentTok{# 1pt}
                            \DataTypeTok{y =}\NormalTok{ Number_of_incidents,  }\CommentTok{# 1pt}
                            \DataTypeTok{fill =}\NormalTok{ Suburb)) }\OperatorTok{+}\StringTok{  }\CommentTok{# 1pt}

\StringTok{  }\KeywordTok{geom_bar}\NormalTok{(}\DataTypeTok{stat =} \StringTok{"identity"}\NormalTok{,   }\CommentTok{# 1pt}
             \DataTypeTok{position =} \StringTok{"dodge"}\NormalTok{) }\OperatorTok{+}\StringTok{  }\CommentTok{# 1pt}
\StringTok{  }\KeywordTok{theme}\NormalTok{(}\DataTypeTok{axis.text.x =} \KeywordTok{element_text}\NormalTok{(}\DataTypeTok{angle =} \DecValTok{90}\NormalTok{, }\DataTypeTok{vjust =} \FloatTok{0.5}\NormalTok{, }\DataTypeTok{hjust=}\DecValTok{1}\NormalTok{)) }\CommentTok{# 1pt}
\end{Highlighting}
\end{Shaded}

\includegraphics{Assignment_1_ETC1010_5510_files/figure-latex/unnamed-chunk-13-1.pdf}

\hypertarget{question-21-change-the-x-and-y-axis-labels-to-years-and-incidents-respectively-for-the-figure-in-question-20-and-use-the-black-and-white-theme}{%
\section{Question 21: Change the x and y-axis labels to ``Years'' and "
Incidents", respectively, for the figure in Question 20 and use the
black and white
theme}\label{question-21-change-the-x-and-y-axis-labels-to-years-and-incidents-respectively-for-the-figure-in-question-20-and-use-the-black-and-white-theme}}

\begin{Shaded}
\begin{Highlighting}[]
\KeywordTok{ggplot}\NormalTok{(comparison_data, }\KeywordTok{aes}\NormalTok{(}\DataTypeTok{x =}\NormalTok{ Year,     }\CommentTok{# 1pt}
                            \DataTypeTok{y =}\NormalTok{ Number_of_incidents,  }\CommentTok{# 1pt}
                            \DataTypeTok{fill =}\NormalTok{ Suburb)) }\OperatorTok{+}\StringTok{   }\CommentTok{# 1pt}
\StringTok{  }\KeywordTok{geom_bar}\NormalTok{(}\DataTypeTok{stat=} \StringTok{"identity"}\NormalTok{,    }\CommentTok{# 1pt}
             \DataTypeTok{position =} \StringTok{"dodge"}\NormalTok{) }\OperatorTok{+}\StringTok{  }\CommentTok{# 1pt}
\StringTok{  }\KeywordTok{theme_bw}\NormalTok{() }\OperatorTok{+}\StringTok{  }\CommentTok{# 1pt}
\KeywordTok{theme}\NormalTok{(}\DataTypeTok{axis.text.x =} \KeywordTok{element_text}\NormalTok{(}\DataTypeTok{angle =} \DecValTok{90}\NormalTok{, }\DataTypeTok{vjust =} \FloatTok{0.5}\NormalTok{, }\DataTypeTok{hjust=}\DecValTok{1}\NormalTok{)) }\OperatorTok{+}\StringTok{  }\CommentTok{# 1pt}
\StringTok{  }\KeywordTok{xlab}\NormalTok{(}\StringTok{"Years"}\NormalTok{) }\OperatorTok{+}\StringTok{  }\CommentTok{# 1pt}
\StringTok{  }\KeywordTok{ylab}\NormalTok{(}\StringTok{"Incidents"}\NormalTok{)  }\CommentTok{# 1pt}
\end{Highlighting}
\end{Shaded}

\includegraphics{Assignment_1_ETC1010_5510_files/figure-latex/unnamed-chunk-14-1.pdf}

\hypertarget{question-22-add-the-following-title-to-the-figure-constructed-in-question-21-number-of-criminal-incidents}{%
\section{Question 22: Add the following title to the figure constructed
in Question 21: ``Number of criminal
incidents''}\label{question-22-add-the-following-title-to-the-figure-constructed-in-question-21-number-of-criminal-incidents}}

\begin{Shaded}
\begin{Highlighting}[]
\KeywordTok{ggplot}\NormalTok{(comparison_data, }\KeywordTok{aes}\NormalTok{(}\DataTypeTok{x =}\NormalTok{ Year,    }\CommentTok{# 1pt}
                            \DataTypeTok{y =}\NormalTok{ Number_of_incidents,  }\CommentTok{# 1pt}
                            \DataTypeTok{fill =}\NormalTok{ Suburb)) }\OperatorTok{+}\StringTok{   }\CommentTok{# 1pt}
\StringTok{  }\KeywordTok{geom_bar}\NormalTok{(}\DataTypeTok{stat=} \StringTok{"identity"}\NormalTok{,    }\CommentTok{# 1pt}
             \DataTypeTok{position =} \StringTok{"dodge"}\NormalTok{) }\OperatorTok{+}\StringTok{  }\CommentTok{# 1pt=}
\StringTok{  }\KeywordTok{theme_bw}\NormalTok{() }\OperatorTok{+}\StringTok{  }\CommentTok{# 1pt}
\StringTok{  }\KeywordTok{theme}\NormalTok{(}\DataTypeTok{axis.text.x =} \KeywordTok{element_text}\NormalTok{(}\DataTypeTok{angle =} \DecValTok{90}\NormalTok{, }\DataTypeTok{vjust =} \FloatTok{0.5}\NormalTok{, }\DataTypeTok{hjust=}\DecValTok{1}\NormalTok{)) }\OperatorTok{+}\StringTok{   }\CommentTok{# 1pt}
\StringTok{  }\KeywordTok{xlab}\NormalTok{(}\StringTok{"Years"}\NormalTok{) }\OperatorTok{+}\StringTok{   }\CommentTok{# 1pt}
\StringTok{ }\KeywordTok{ylab}\NormalTok{(}\StringTok{"Incidents"}\NormalTok{) }\OperatorTok{+}\StringTok{   }\CommentTok{# 1pt}
\StringTok{  }\KeywordTok{ggtitle}\NormalTok{(}\StringTok{"Number of criminal incidents"}\NormalTok{)  }\CommentTok{# 1pt}
\end{Highlighting}
\end{Shaded}

\includegraphics{Assignment_1_ETC1010_5510_files/figure-latex/unnamed-chunk-15-1.pdf}

\hypertarget{question-23-by-using-facet_wrap-create-a-line-plot-to-show-the-trends-for-number_of_incidents-for-each-of-the-three-suburbs}{%
\section{Question 23: By using ``facet\_wrap'', create a line plot to
show the trends for ``Number\_of\_incidents'' for each of the three
suburbs}\label{question-23-by-using-facet_wrap-create-a-line-plot-to-show-the-trends-for-number_of_incidents-for-each-of-the-three-suburbs}}

\begin{Shaded}
\begin{Highlighting}[]
\KeywordTok{ggplot}\NormalTok{(comparison_data, }\KeywordTok{aes}\NormalTok{(}\DataTypeTok{x =}\NormalTok{ Year,   }\CommentTok{# 1pt}
                            \DataTypeTok{y =}\NormalTok{ Number_of_incidents,  }\CommentTok{# 1pt}
                            \DataTypeTok{fill =}\NormalTok{Suburb)) }\OperatorTok{+}\StringTok{  }\CommentTok{# 1pt}
\StringTok{  }\KeywordTok{geom_line}\NormalTok{() }\OperatorTok{+}\StringTok{   }\CommentTok{# 1pt}
\StringTok{  }\KeywordTok{facet_wrap}\NormalTok{(}\OperatorTok{~}\NormalTok{Suburb) }\OperatorTok{+}\StringTok{  }\CommentTok{# 1pt=}
\StringTok{  }\KeywordTok{theme_bw}\NormalTok{() }\OperatorTok{+}\StringTok{  }\CommentTok{# 1pt}
\StringTok{  }\KeywordTok{theme}\NormalTok{(}\DataTypeTok{axis.text.x =} \KeywordTok{element_text}\NormalTok{(}\DataTypeTok{angle =} \DecValTok{90}\NormalTok{, }\DataTypeTok{vjust =} \FloatTok{0.5}\NormalTok{, }\DataTypeTok{hjust=}\DecValTok{1}\NormalTok{))  }\CommentTok{# 1pt}
\end{Highlighting}
\end{Shaded}

\includegraphics{Assignment_1_ETC1010_5510_files/figure-latex/unnamed-chunk-16-1.pdf}

\hypertarget{question-24-transform-the-data-set-named-comparison_data-into-a-wide-format-where-the-suburbs-of-coogee-redfern-and-zetland-are-displayed-as-columns}{%
\section{\texorpdfstring{Question 24: Transform the data set named
\emph{comparison\_data} into a wide format where the suburbs of Coogee,
Redfern, and Zetland are displayed as
columns}{Question 24: Transform the data set named comparison\_data into a wide format where the suburbs of Coogee, Redfern, and Zetland are displayed as columns}}\label{question-24-transform-the-data-set-named-comparison_data-into-a-wide-format-where-the-suburbs-of-coogee-redfern-and-zetland-are-displayed-as-columns}}

\begin{Shaded}
\begin{Highlighting}[]
\NormalTok{comparison_data }\OperatorTok
\StringTok{  }\KeywordTok{pivot_wider}\NormalTok{(}\DataTypeTok{id_cols =} \OperatorTok{-}\NormalTok{Number_of_incidents,   }\CommentTok{# 1pt}
              \DataTypeTok{names_from =}\NormalTok{ Suburb,  }\CommentTok{# 1pt}
              \DataTypeTok{values_from =}\NormalTok{Number_of_incidents)  }\CommentTok{# 1pt}
\end{Highlighting}
\end{Shaded}

\begin{verbatim}
## # A tibble: 10 x 4
##     Year Coogee Redfern Zetland
##    <dbl>  <dbl>   <dbl>   <dbl>
##  1  2010    897    3225     197
##  2  2011   1189    3822     318
##  3  2012    877    3959     380
##  4  2013    885    4440     312
##  5  2014    762    4400     359
##  6  2015    912    4674     562
##  7  2016   1016    5623     493
##  8  2017   1011    4411     526
##  9  2018   1013    4102     572
## 10  2019   1119    4052     621
\end{verbatim}

\end{document}
